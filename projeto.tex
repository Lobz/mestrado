\documentclass[twoside,12pt,a4paper]{report}
\usepackage[english,brazilian]{babel}
\usepackage{lmodern}
\usepackage{ifxetex,ifluatex}
\usepackage{fixltx2e} % provides \textsubscript
\ifnum 0\ifxetex 1\fi\ifluatex 1\fi=0 % if pdftex
  \usepackage[T1]{fontenc}
  \usepackage[utf8]{inputenc}
\else % if luatex or xelatex
  \ifxetex
    \usepackage{mathspec}
  \else
    \usepackage{fontspec}
  \fi
  \defaultfontfeatures{Ligatures=TeX,Scale=MatchLowercase}
\fi
\usepackage[twoside,margin=1.2in]{geometry}
\usepackage{graphicx}
\graphicspath{ {figuras/} }
\usepackage{amsmath}
\usepackage{amsfonts}
\usepackage[square,sort]{natbib}
\usepackage{titlesec}
\usepackage{verbatim}
\usepackage[textfont={scriptsize}]{caption}
\setlength{\topmargin}{0cm}
\titleformat{\paragraph}
{\normalfont\normalsize\bfseries}{\theparagraph}{1em}{}
\titlespacing*{\paragraph}
{0pt}{3.25ex plus 1ex minus .2ex}{1.5ex plus .2ex}
\usepackage{appendix}
\usepackage[final]{pdfpages}

\begin{document}

\title{Mecanismos de determinação da distribuição espacial e coexistência em florestas tropicais:
filtragem de nicho, dispersão e facilitação}
\author{Mali Oz Cenamo Salles}

%\maketitle

\begin{abstract}

A análise da distribuição espacial dos indivíduos de uma comunidade vegetal tropical pode dar muita
informação sobre os processos que organizam essa comunidade. Dados de parcelas permanentes mostram
que um grande número de pares de espécies de árvores tem associação espacial positiva. Neste
trabalho, investigaremos associações espaciais positivas entre espécies vegetais, e compararemos
três mecanismos possíveis causadores dessas associações espaciais: filtragem ambiental sob uma
dinâmica neutra, interação ecológica positiva (facilitação) e co-dispersão (dispersão para os mesmos
ambientes).

\end{abstract}

\section{Introdução}

Uma questão fundamental em ecologia de comunidades é o que determina a riqueza (quantidade de
espécies)da comunidade. 
A riqueza pode ser muito diferente de uma comunidade para outra, sendo que as comunidades com
maior riqueza estão na região tropical.
No Brasil encontram-se florestas tropicais extremamente diversas, com onde centenas de
espécies de árvores coexistem em cada hectar. Portanto, temos ambientes propícios ao estudo dos mecanismos
que permitem a coexistência de muitas espécies em uma mesma comunidade.

Alguns dos processos mais estudados associados à coexistência de espécies são repartição de
nicho \citep{refs}, limitação de dispersão \citep{refs} e dependência negativa de densidade sobrecompensante
(processos de Janzen-Connel)\citep{refs}. 
Esses processos garantem que uma população limite seu próprio crescimento mais do que limita o
crescimento de populações de outras espécies, o que limita a possibilidade de exclusão
competitiva. 
Esses processos também são espacialmente
estruturados, produzindo padrões espaciais distintos. Por exemplo, processos de
Janzen-Connel impossibilitam grandes agregações de indivíduos de uma mesma espécie,
produzindo um padrão espacial regular, enquanto limitação de dispersão faz com que
indivíduos ocorram próximos de seus coespecíficos, gerando um padrão agregado.
Assim, a distribuição espacial dos indivíduos de uma comunidade vegetal pode dar muita
informação sobre os processos que organizam essa comunidade. 

Porém, diferentes processos podem gerar padrões espaciais semelhantes. Por exemplo, se o
fator mais relevante na estrutura da comunidade é a repartição de nicho, então cada espécie
deve ocorrer em manchas de habitat favorável, e portanto o padrão espacial também será
agregado. Em outras palavras, o mesmo padrão pode ser gerado por diferentes processos. Além
disso, é possível que diversos processos estejam ocorrendo simultaneamente em uma comunidade.

Um processo que não é tão estudado quanto os processos acima é a facilitação.
“Facilitação” é o termo usado em ecologia para se referir às interações ecológicas positivas entre
organismos do mesmo nível trófico \citep{Pakeman2009}. 
Através da facilitação, uma espécie pode beneficiar o desempenho de outra. 
A facilitação pode permitir que uma
espécie ocupe um ambiente no qual ela não se estabeleceria por conta própria \citep{Lortie2004},
ou acelerar a colonização de um novo ambiente (inclusive por espécies invasoras, v.
\citep{Wundrow2012}, ou aumentar a abundância de uma espécie em determinada comunidade \citep{Alados2006; CallawayBook}.
Como a facilitação faz com que certas espécies se beneficiem da coexistência com certas
outras, ela é um mecanismo que propicia a coexistência dessas espécies.
A facilitação também produz um padrão espacial particular, onde a espécie facilitada tende a
ocorrer espacialmente associada à espécie facilitadora (refs).
Porém, outros processos podem produzir esse mesmo padrão. Por exemplo, a filtragem de nicho
também pode produzir associação entre espécies, se essas espécies tiverem alta sobreposição
de nicho e por isso estiverem restritas às mesmas manchas de hábitat.

Outro processo ecológico que pode resultar em associação espacial entre espécies é a
dispersão de sementes pelos mesmos agentes dispersores. Espécies com a mesma síndrome de
dispersão podem oferecer diásporos aos mesmos frugívoros, que potencialmente carregariam as
sementes para os mesmos locais, levando a uma associação espacial das sementes. 

Outra teoria muito estudada é a Teoria Neutra, que prevê que se as populações limitam eu
próprio crescimento exatamente na mesma medida que os das outras populações, então a
composição da comunidade será aleatória, e a diversidade da comunidade dependerá de taxas de
imigração e especiação. A teoria neutra não pressupõe que haja diferenças ecológicas entre
as espécies, como a teoria de nicho, nem que haja processos que ocorram apenas em densidades
altas; assim, é um modelo simples que pode ser usado como um modelo nulo contra o qual comparar
todos os outros. O modelo neutro não é originalmente espacializado, e portanto, não há
um padrão espacial associado a ele; entretanto, suas premissas podem ser usadas de base para
criar modelos espacializados simples que podem servir como modelo nulo. Além disso, esses
modelos podem ser incrementados para incluir outros processos de forma controlada, ajudando a
gerar previsões de que padrões seriam gerados por quais processos.

\subsubsection{Previsões}

O processo de filtragem ambiental deve gerar um padrão de agregação intraespecífica, já que os
indivíduos de uma mesma espécie devem ocorrer no mesmo microhabitat. Além disso, prevemos encontrar
correlação entre a presença de indivíduos dessa espécie e um conjunto de intervalos de valores de
variáveis abióticas, que representariam seu nicho fundamental \citep{Hutchinson1957}. Se a
agregação espacial entre duas espécies é causada por uma adaptação às mesmas condições ambientais,
devemos encontrá-las associadas às mesmas variáveis.

O processo de facilitação deve gerar um padrão de agregação da espécie facilitada em torno na
espécie facilitadora. Não há previsão de que a facilitadora esteja agregada (exceto pelo efeito de
outros mecanismos). Porém, as variáveis ambientais também podem ser alteradas localmente pelos
organismos presentes, por exemplo através de sombreamento, associação de micorriza, ou
serrapilheira; de fato, as formas mais comuns de facilitação são através de alteração de
características ambientais \citep{TODO,refs}. Testaremos alguns métodos para distingüir entre simples
filtragem ambiental e filtragem ambiental intermediada por facilitação: primeiramente, esperamos que
no caso de facilitação a associação entre as espécies seja mais forte do que esperado apenas
levando-se em conta a associação delas com as variáveis ambientais, então faremos um modelo de
simulação da distribuição espacial dos indivíduos sob filtragem ambiental para usar como modelo
nulo.

O processo de co-dispersão deve causar um padrão de agregação na chuva de sementes. Podemos avaliar
a relevância do padrão de dispersão da chuva de sementes com o padrão de dispersão dos indivíduos
estabelecidos; esperamos que haja associação entre adultos e sementes de uma mesma espécie, e que no
caso de co-dispersão, a associação entre sementes de espécies diferentes seja mais forte do que o
esperado considerando a associação entre os adultos da mesma espécie.

Um objetivo deste trabalho é testar a hipótese de que facilitação é um mecanismo importante de
coexistência em uma formação florestal tropical sujeita à alagamento (Floresta de Restinga). Para
isso, perguntamos: (1) há associação espacial entre populações de diferentes espécies? (2) Para
populações associadas, os indivíduos têm distribuição agregada e associada a variáveis ambientais?
(3) Nesse caso, existe associação das variáveis ambientais a uma das espécies? (4) Para essas mesmas
populações, a associação entre as espécies ocorre desde a chuva de sementes? Esperamos que, se
facilitação for um processo ecológico importante, ocorra associação espacial entre populações de
diferentes espécies e que essa associação não seja explicada pelas características do ambiente e da
chuva de sementes, ou que a associação seja assimétrica, e que as variáveis ambientais estejam
alteradas por uma espécie.

A tese será dividida em dois capítulos que abordam diferentes processos ecológicos levando à
associação espacial de populações de diferentes espécies. No primeiro capítulo, analisaremos o
padrão espacial da chuva de sementes em comparação com a distribuição dos adultos, de acordo com a
teoria de que a distribuição dos adultos esteja determinada pela chuva de sementes. No segundo
capítulo, abordaremos a relação espacial entre populações de diferentes espécies e com variáveis
ambientais, testando a hipótese da associação das espécies com características ambientais conforme a
Teoria de Nicho.


\section{Justificativa}

O padrão espacial de uma comunidade permite importantes inferências a respeito dos processos que
estruturam a comunidade e permitem a coexistência de um grande número de espécies, e há atualmente
um grande número de projetos ao redor do mundo mapeando parcelas de florestas para compreender a
estrutura da diversidade. Entretanto, o mesmo padrão pode ser gerado por uma grande variedade de
processos. Este projeto visa avançar a metodologia de diferenciação dos processos por trás de
padrões espaciais, além de aplicar esses métodos para compreender os principais mecanismos
determinando a composição da comunidade da Floresta de Restinga. Diversos estudos usam a facilitação
como estratégia de restauração de florestas tropicais \citep{Zwiener2014,ref}, e o
conhecimento sobre formas de facilitação pode ajudar a propor novas estratégias.

\section{Área de estudo}
A importância de facilitação é conhecida especialmente em ambientes com alto estresse
ambiental, onde uma espécie mais tolerante ao estresse pode amenizar o estresse ambiental e
permitir a sobrevivência de uma espécie mens tolerante \citep{BertnessHacker1994,
CallawayHacker1995}. 
A facilitação tem sido descrita em uma grande variedade de ambientes, em
uma variedade de formas \citep{McIntire2014}, mas o fenômeno é mais estudado em ambientes com
condições extremas, especialmente em gradientes de estresse, e na facilitação como redução do
estresse abiótico \citep{Brooker2008}. Em florestas, a facilitação é freqüentemente
estudada no contexto da sucessão ecológica \citep{refs refs refs}, mas recentemente, alguns
estudos têm encontrado evidências de que a facilitação é importante para a composição de
comunidades de florestas tropicais
\citep{Ledo2015,refs}. A Floresta de Restinga é um ambiente propício para detecção de facilitação devido ao alto estresse
ambiental, dado o solo arenoso e pobre em nutrientes, e a ocorrência de inundações sazonais de água
salgada ou salobra \citep{TODO refs}. 


A Mata Atlântica tem apenas 11,4 a 16\% de sua área original restantes \citep{Ribeiro2009}, e
devido a sua alta riqueza de espécies endêmicas, a mata ao longo da costa é classificada como um dos
hotspots para conservação da biodiversidade \citep{Myers2000}. Boa parte de sua área preservada
está no Parque Estadual da Serra do Mar. A Floresta de Restinga é uma formação florestal que ocorre
de 0 a 50m de altitude, próxima ao mar, sobre cordões arenosos \citep{Joly2008}. No presente
estudo, os dados foram e serão coletados em uma parcela permanente de 1 ha de Floresta de Restinga
do Núcleo Picinguaba do Parque Estadual da Serra do Mar, instalada no âmbito do Projeto Temático
“Gradiente Funcional” (Biota/FAPESP 03/12595-7) e subdividida em 100 subparcelas de 10 m x 10 m. A
parcela se localiza próxima à Praia da Fazenda, Ubatuba-SP, a cerca de 10m de altitude, e parte de
sua área é alagada periodicamente.

Possíveis mecanismos de facilitação nesse ambiente incluiriam sombreamento
\citep{Castanho2014, etc}, enriquecimento de solo através de serrapilheira (achar refs, atração de
organismos dispersores \citep{achar refs CallawayBook} ou de organismos que controlem a ação de
herbívoros \citep{refs}, associações benéficas com micorriza \citep{Simard1997}, proteção
estrutural contra estresse mecânico \citep{refs}, fusão de fustes \citep{McIntire2011}, entre
outros. Facilitação através de micorriza foi demonstrada em \citep{Simard1997} e a presença de
micorriza pode aumentar a tolerância de espécies arbóreas à inundação \citep{Fougnies2007}. 

\section{Análises de dados}

Para analisar os dados de padrão de pontos, usaremos como referência \citep{IllianBook} e
\citep{WiegandBook}, usando principalmente a função de associação espacial de pontos bivariada
$g_{ab}(r)$ \citep{IllianBook,WiegandBook,Stoyan1994}. Essa função, também chamada de medida de
anéis em O, representa o número de indivíduos do grupo $b$ presentes na classe de distância $r$ em
torno de indivíduos do grupo $a$, normalizada em relação ao valor esperado da função para uma
distribuição uniforme (aleatória). Dessa forma, $g_{ab}(r) > 1$ significa que há mais indivíduos a
espécie $b$ a um raio $r$ de invidívuos da espécie $a$ do que o esperado,
enquanto $g_{ab}(r) < 1 $ significa que o número de indivíduos nessa classe de distâncias é menor
que o esperado. A estatística $g_{ab}(r)$ dá informação sobre associação e dissociação espacial
entre dois grupos de indivíduos em várias escalas de distância, de forma que podemos obter um padrão
misto, como dissociação em curtas distâncias e associação em distâncias maiores, ou vice versa. O
raio máximo para o qual podemos usar essa estatística é igual a metade da menor dimensão da parcela
onde os indivíduos foram mapeados; neste caso, 50m \citep{IllianBook}. Neste trabalho usaremos essa
função para avaliar o padrão de associação espacial entre pares de espécies, e também o padrão de
agregação ou desagregação de uma espécie, expresso pela função de associação de uma espécie com ela
mesma ($g_{aa}(r)$).

Para avaliar a significância dos padrões encontrados, faremos simulações computacionais de padrões
de pontos para representar o modelo nulo de padrão aleatório. Cada simulação gerará um valor de
$g$, que será usado para gerar um envelope de confiança e uma estatística de significância
\citep{IllianBook,DiggleBook}. Porém, como estaremos aplicando essa estatística em toda uma
comunidade, a análise de significância par-a-par pode levar a uma alta probabilidade de erro tipo 1.
Por isso, farei uma revisão da literatura em busca de métodos aplicáveis para um grande número de
espécies e formas de controlar esse erro. Além disso, usarei uma simulação de um modelo neutro
espacialmente explícito para comparar, entre o modelo neutro e os dados reais, a quantidade de pares de espécies para os quais a estatística
$g$ é significativa.

As variáveis ambientais foram coletadas por \cite{Kelly} em cada uma das subparcelas de 10mx10m.
Para comparar a distribuição espacial dos indivíduos com as variáveis ambientais, usaremos o dado de
densidade de indivíduos de cada espécie em cada parcela. Parte deste trabalho será estudar os
melhores modelos para prever a densidade das espécies a partir das variáveis ambientais, e
selecionar o melhor modelo para cada espécie usando AIC (v. \cite{Bagchi2011,Henrys2009}). A partir
dessa análise da densidade, poderemos simular a distribuição dos indivíduos de cada espécie
determinada apenas pelas variáveis ambientais usando processos de Poisson heterogêneos
\citep{WiegandBook}, com a função de densidade determinada pelo modelo selecionado. Usaremos essas
simulações como um segundo modelo para calcular a estatística $g$, para avaliar se a
estatística $g$ é prevista pela associação das espécies com variáveis ambientais, ou se há uma
associação ou dissociação significativa entre espécies que não é explicada pelas variáveis
ambientais.

A chuva de sementes está sendo coletada mensalmente em 60 coletores circulares dispostos
aproximadamente no centro de subparcelas que foram selecionadas aleatoriamente dentre as 100
disponíveis. As sementes serão identificadas e classificadas de acordo com a síndrome de dispersão.
Usarei modelos lineares generalizados para medir a influência da proximidade de adultos da mesma
espécie na densidade de sementes em cada coletor \citep{Hardesty2002}. Também determinarei se há
associação espacial entre as
chuvas de sementes de espécies diferentes usando SADIE (Spatial Analysis by Distance Indices –
\cite{Perry1995}), uma técnica que integra a estrutura espacial das amostras no cálculo da
correlação \citep{Li2012}. Caso encontre associação, tentarei determinar se ela pode ser prevista pela presença dos adultos (usando as densidades previstas pelos
modelos lineares) ou pela síndrome de dispersão.

\section{Cronograma}

	\begin{figure}[ht]
		\centering \includegraphics[width=16cm]{cronograma.png}
    \end{figure}

A qualificação no Programa de Pós-Graduação em Ecologia da UNICAMP está prevista para o final do
primeiro ano. Ao longo dos três anos, serão feitos três relatórios (dois parciais e um final),
cursadas disciplinas de interesse e realizados estágios de docência. Também participarei de
congressos, onde apresentarei trabalhos resultantes deste projeto.

\section{Resultados esperados}
O presente estudo contribuirá para o entendimento dos fatores que determinam a distribuição espacial
das espécies arbóreas em florestas tropicais, e para o conhecimento das interações ecológicas entre
espécies de árvores, resultando em última instância em avanço do conhecimento sobre a diversidade
das florestas tropicais. Durante o projeto serão redigidos dois artigos, e o trabalho será apresentado
em congressos.



    \bibliographystyle{apalike}
    \bibliography{tcc}
\end{document}
