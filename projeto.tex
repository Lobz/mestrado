\documentclass[twoside,12pt,a4paper]{report}
\usepackage[english,brazilian]{babel}
\usepackage{lmodern}
\usepackage{ifxetex,ifluatex}
\usepackage{fixltx2e} % provides \textsubscript
\ifnum 0\ifxetex 1\fi\ifluatex 1\fi=0 % if pdftex
  \usepackage[T1]{fontenc}
  \usepackage[utf8]{inputenc}
\else % if luatex or xelatex
  \ifxetex
    \usepackage{mathspec}
  \else
    \usepackage{fontspec}
  \fi
  \defaultfontfeatures{Ligatures=TeX,Scale=MatchLowercase}
\fi
\usepackage[twoside,margin=1.2in]{geometry}
\usepackage{graphicx}
\graphicspath{ {figuras/} }
\usepackage{amsmath}
\usepackage{amsfonts}
\usepackage[square,sort]{natbib}
\usepackage{titlesec}
\usepackage{verbatim}
\usepackage[textfont={scriptsize}]{caption}
\setlength{\topmargin}{0cm}
\titleformat{\paragraph}
{\normalfont\normalsize\bfseries}{\theparagraph}{1em}{}
\titlespacing*{\paragraph}
{0pt}{3.25ex plus 1ex minus .2ex}{1.5ex plus .2ex}
\usepackage{appendix}
\usepackage[final]{pdfpages}

\begin{document}

\title{Mecanismos de determinação da distribuição espacial e coexistência em florestas tropicais:
filtragem de nicho, dispersão e facilitação}
\author{Mali Oz Cenamo Salles}

%\maketitle

\begin{abstract}

A análise da distribuição espacial dos indivíduos de uma comunidade vegetal tropical pode dar muita
informação sobre os processos que organizam essa comunidade. Dados de parcelas permanentes mostram
que um grande número de pares de espécies de árvores tem associação espacial positiva. Neste
trabalho, investigaremos associações espaciais positivas entre espécies vegetais, e compararemos
três mecanismos possíveis causadores dessas associações espaciais: filtragem ambiental sob uma
dinâmica neutra, interação ecológica positiva (facilitação) e co-dispersão (dispersão para os mesmos
ambientes).

\end{abstract}

\section{Introdução}

Uma questão fundamental em ecologia de comunidades é o que determina a riqueza de
espécies em uma comunidade. 
A riqueza pode ser muito diferente de uma comunidade para outra, sendo que as comunidades com
maior riqueza estão na região tropical.
No Brasil encontram-se florestas tropicais extremamente diversas, com onde centenas de
espécies de árvores coexistem em cada hectar. Portanto, temos ambientes propícios ao estudo dos mecanismos
que permitem a coexistência de muitas espécies em uma mesma comunidade.

Alguns dos processos mais estudados associados à coexistência de espécies são repartição de
nicho \citep{Hutchinson1957}, limitação de dispersão \citep{HurttPacala1995} e dependência
negativa de densidade (DND)
\citep{refs}. 
Na repartição de nicho, cada espécie se especializa em certos recursos ou condições
ambientais, o que limita a competição entre as espécies. Já a limitação de dispersão gera
agregação dos coespecíficos, o que faz com que a maior competição seja intraespecífica; e na
DND, a densidade de cada população controla o crescomento populacional.
Assim, esses processos garantem que uma população limite seu próprio crescimento mais do que limita o
crescimento de populações de outras espécies, o que limita a possibilidade de exclusão
competitiva.
Esses processos também são espacialmente
estruturados, produzindo padrões espaciais distintos. Por exemplo, dependência negativa de
densidade impossibilita grandes agregações de indivíduos de uma mesma espécie,
produzindo um padrão espacial regular, enquanto limitação de dispersão faz com que
indivíduos ocorram próximos de seus coespecíficos, gerando um padrão agregado.
Assim, a distribuição espacial dos indivíduos de uma comunidade vegetal pode dar muita
informação sobre os processos que organizam essa comunidade.  

Porém, diferentes processos podem gerar padrões espaciais semelhantes. Por exemplo, se o
fator mais relevante na estrutura da comunidade é a repartição de nicho, então cada espécie
deve ocorrer em manchas de habitat favorável, e portanto o padrão espacial também será
agregado. Em outras palavras, o mesmo padrão pode ser gerado por diferentes processos. Além
disso, é possível que diversos processos estejam ocorrendo simultaneamente em uma comunidade.
Por causa disso, inferir quais processos ocorrem numa comunidade a partir da distribuição
espacial dos indivíduos não é uma tarefa simples.

Um processo que não é tão estudado quanto os processos acima é a facilitação.
“Facilitação” é o termo usado em ecologia para se referir às interações ecológicas positivas entre
organismos do mesmo nível trófico \citep{Pakeman2009}. 
Através da facilitação, uma espécie pode beneficiar o desempenho de outra. 
A facilitação pode permitir que uma
espécie ocupe um ambiente no qual ela não se estabeleceria por conta própria \citep{Lortie2004},
ou acelerar a colonização de um novo ambiente (inclusive por espécies invasoras, v.
\citep{Wundrow2012}, ou aumentar a abundância de uma espécie em determinada comunidade \citep{Alados2006}.
A facilitação tem sido descrita em uma grande variedade de ambientes, em
uma variedade de formas \citep{McIntire2014}, mas o fenômeno é mais estudado em ambientes com
alto estresse ambiental, onde uma espécie mais tolerante ao estresse pode amenizar o estresse ambiental e
permitir a sobrevivência de uma espécie mais sucetível \citep{BertnessHacker1994,
BertnessCallaway1994,Brooker2008}. Em geral, a relação é assimétrica, com a espécie tolerante
sofrendo competição da espécie sucetível. Porém, quando a facilitação é importante para o
desempenho da espécie sucetível, ela se torna incapaz de excluir competitivamente a outra, já
que seu desempenho é reduzido quando a espécie tolerante é rara. Dessa forma, a facilitação
mesmo que assimétrica é reconhecida como um mecanismo de coexistência.
Em florestas, a facilitação é freqüentemente
estudada no contexto da sucessão ecológica \citep{refs refs refs}, mas também há evidências de que a facilitação é importante para a composição de
comunidades de florestas tropicais maduras
\citep{Ledo2015}.
A facilitação também produz um padrão espacial particular, onde a espécie facilitada tende a
ocorrer espacialmente associada à espécie facilitadora.
Porém, outros processos podem produzir esse mesmo padrão. Por exemplo, a filtragem de nicho
também pode produzir associação entre espécies, se essas espécies tiverem alta sobreposição
de nicho e por isso estiverem restritas às mesmas manchas de hábitat. Isso não quer dizer que
os padrões gerados por esses dois processos sejam idênticos. A facilitação é frequentemente
um processo assimétrico, isto é, uma espécie facilita a outra, mas não recebe facilitação de
volta. Então, associação espacial resultante de facilitação também será assimétrica, enquanto
a associação resultante de filtragem de nicho será simétrica. Além disso, a associação
espacial resultante de filtragem de nicho depende de heterogeneidade ambiental, então se
restringirmos nossa análise a locais homogêneos, por exemplo, a uma mancha de hábitat, o
padrão deve desaparecer. Isso também significa que esses processos podem gerar padrões com
escalas diferentes, já que a escala de associação por filtragem de nicho está ligada à escala
da heterogeidade ambiental, enquanto a escala da associação por facilitação está ligada à
escala das interações ecológicas entre as espécies. Por fim, o processo de filtragem de nicho
também pode ser detectado através da análise da correlação entre densidades populacionais e
variáves ambientais, já que cada população ficaria restrita aos locais com características
favoráveis à sua espécie.

Outro processo ecológico que pode resultar em associação espacial entre espécies é a
dispersão de sementes pelos mesmos agentes dispersores. Espécies com a mesma síndrome de
dispersão e que frutifiquem na mesma época podem oferecer diásporos aos mesmos frugívoros, que carregariam as
sementes para os mesmos locais, levando a uma associação espacial das sementes. Se as
sementes estiverem associadas espacialmente, e nenhum outro efeito espacial relevante ocorrer
sobre esses indivíduos, então é provável que os adultos gerados a partir dessas sementes ainda estejam
associados. Assim, a associação espacial de sementes pode levar a uma associação espacial de
adultos. Note-se que a associação espacial de sementes pode também ser causada por uma
associação espacial dos adultos atuais, já que as sementes são produzidas por adultos. A
limitação de dispersão, que mantém os indivíduos de uma população agregados, também mantém
aproximadamente a configuração espacial da comunidade, de forma que espécies associadas
provavelmente terão uma chuva de semente ainda associada. Porém, uma dispersão aleatória,
mesmo que limitada,
tenderia a espalhar as sementes de cada espécie independentemente e assim diminuir o grau de
associação entre elas. Assim, se a dispersão das duas espécies for independente, é esperado
logicamente que ao longo das
gerações o padrão de associação ou dissociação espacial se dissolva. Isso significa que para
a associação das sementes resultar em aumento da associação espacial dos adultos, a
associação espacial das sementes deve ser maior do que aquela produzida por simples limitação
de dispersão.

Na descrição dos processos acima, recorri com frequência à consideração do que cada processo
causaria ``na ausência de outros fatores''. O próximo passo seria nos perguntarmos qual seria
o padrão gerado na ausência de {\em todos} esses fatores. Uma teoria fundamental para essa
consideração é a teoria neutra \citep{Hubbell1979}.
A teoria neutra prevê que se as populações limitam eu
próprio crescimento exatamente na mesma medida que os das outras populações, então a
composição da comunidade será aleatória, e a diversidade da comunidade dependerá de taxas de
imigração e especiação \citep{Hubbell2001}. A teoria neutra não pressupõe que haja diferenças ecológicas entre
as espécies, como a teoria de nicho, nem que haja processos que ocorram apenas em densidades
altas; assim, é um modelo simples que pode ser usado como um modelo nulo contra o qual comparar
todos os outros. O modelo neutro não é originalmente espacializado, e portanto, não há
um padrão espacial associado a ele; entretanto, suas premissas podem ser usadas como base para
criar modelos espacializados simples que podem servir como modelo nulo. Além disso, esses
modelos podem ser incrementados para incluir outros processos de forma controlada, ajudando a
gerar previsões de que padrões seriam gerados por quais processos.

Com a combinação de simulações e análise dos dados, poderemos inferir quais dos processos
propostos são relevantes para a configuração da comunidade estudada. Com essa informação,
poderemos discutir se são esses processos que determinam a riqueza da comunidade.
A tese será dividida em três capítulos, o primeiro focado na criação de simulações a partir
do modelo neutro, e os demais abordando diferentes processos ecológicos que levam à
associação espacial de populações de diferentes espécies. No segundo capítulo, analisaremos o
padrão espacial da chuva de sementes em comparação com a distribuição dos adultos, de acordo com a
teoria de que a distribuição dos adultos esteja determinada pela chuva de sementes. No
terceiro
capítulo, abordaremos a relação espacial entre populações de diferentes espécies e com variáveis
ambientais, testando a hipótese da associação das espécies com características ambientais conforme a
teoria de nicho, e a hipótese de associação espacial produzida por facilitação.'
Na conclusão, discutiremos quais processos foram encontrados nas análises, e qual a
importância desses processos para a riqueza da comunidade.


\section{Justificativa}

O padrão espacial de uma comunidade permite importantes inferências a respeito dos processos que
estruturam a comunidade e permitem a coexistência de um grande número de espécies, e há atualmente
um grande número de projetos ao redor do mundo mapeando parcelas de florestas para
compreender os processos ecológicos que ocorrem nessas comunidades. Entretanto, o mesmo padrão pode ser gerado por uma grande variedade de
processos. Este projeto visa identificar quais processos estão ocorrendo em uma comunidade, e
melhor entender como esses processos estruturam espacialmente as populações afetadas. 
O processo de facilitação ainda é pouco estudado em florestas tropicais, e por
isso as evidências de sua ocorrência são raras, e importantes para a ecologia de comunidades.
O impacto da dispersão pelos mesmos dispersores na estrutura espacial das populações não é
normalmente discutido na literatura, e pode determinar a co-ocorrêcia das espécies, tendo
conseqüências para a interação entre elas e sua coexistência. A filtragem de nicho e a
limitação de dispersão são processos muito estudados, mas ainda há muito o que se entender
sobre como esses processos ocorrem em cada comunidade. Em particular, não sabemos como esses
processos funcionam na floresta de restinga.

\section{Área de estudo}
A Mata Atlântica tem apenas 11,4 a 16\% de sua área original restantes \citep{Ribeiro2009}, e
devido a sua alta riqueza de espécies endêmicas, a mata ao longo da costa é classificada como um dos
hotspots para conservação da biodiversidade \citep{Myers2000}. Boa parte de sua área preservada
está no Parque Estadual da Serra do Mar. A Floresta de Restinga é uma formação florestal que ocorre
de 0 a 50m de altitude, próxima ao mar, sobre cordões arenosos \citep{Joly2008}. No presente
estudo, os dados foram e serão coletados em uma parcela permanente de 1 ha de Floresta de Restinga
do Núcleo Picinguaba do Parque Estadual da Serra do Mar, instalada no âmbito do Projeto Temático
“Gradiente Funcional” (Biota/FAPESP 03/12595-7) e subdividida em 100 subparcelas de 10 m x 10 m. A
parcela se localiza próxima à Praia da Fazenda, Ubatuba-SP, a cerca de 10m de altitude, e parte de
sua área é alagada periodicamente.

A floresta de restinga é um ambiente propício para detecção de facilitação devido ao alto estresse
ambiental, dado o solo arenoso e pobre em nutrientes, e a ocorrência de inundações sazonais de água
salgada ou salobra \citep{TODO refs}. 
Possíveis mecanismos de facilitação nesse ambiente incluiriam sombreamento
\citep{Castanho2014}, enriquecimento de solo através de serrapilheira (achar refs, atração de
organismos dispersores \citep{achar refs CallawayBook} ou de organismos que controlem a ação de
herbívoros \citep{refs}, associações benéficas com micorriza \citep{Simard1997} (talvez
relacionada à sobrevivência à inundação\citep{Fougnies2007}), proteção
estrutural contra estresse mecânico \citep{refs}, fusão de fustes \citep{McIntire2011}, entre
outros.

A inundação também configura heterogeneidade ambiental bem definida, o que torna o local
propício para estudar filtragem ambiental.

\section{Coleta de Dados}

A chuva de sementes está sendo coletada mensalmente em 60 coletores circulares dispostos
aproximadamente no centro de subparcelas que foram escolhidas aleatoriamente dentre as 100
disponíveis. As sementes serão identificadas e classificadas de acordo com a síndrome de
dispersão, que já foi determinada para todas as espécies da parcela \citep{martins2014}.

\section{Análises de dados}

Para determinar quais espécies apresentam populações com associação espacial usaremos a função de correlação
de pares bivariada
$g_{12}(r)$ \citep{Stoyan1994}. Essa função representa o
número de indivíduos do padrão de pontos 2 dentro de um anel de raio $r$ e largura $dr$, 
normalizada em relação ao valor esperado da função para um determinado modelo nulo (para o
modelo nulo de distribuição uniforme, a normalização consiste em dividir pela intensidade
$\lambda$ do padrão de pontos do padrão 2 na área de estudo). Dessa forma, $g_{12}(r) > 1$ significa que há mais indivíduos a
espécie 2 a uma distância $r$ de indivíduos da espécie 1 do que o esperado segundo o modelo
nulo, e $g_{12}(r) < 1 $ significa que o número de indivíduos da espécie 2 a distância $r$ de
indivíduos da espécie 1 é menor
que o esperado. A estatística $g_{12}(r)$ dá informação sobre a relação espacial
entre dois padrões de pontos em várias escalas espaciais, de forma que podemos obter um padrão
misto, como dissociação em curtas distâncias e associação em distâncias maiores, ou vice
versa \citep{WiegandMoloney2004}. Por isso, é considerada a estatística de segunda
ordem mais informativa ára descrever padrões de pontos \citep{IllianBook}.  
Para avaliar a simetria da associação espacial, calcularemos tanto
$g_{12}(r)$ quanto $g_{21}(r)$, e compararemos os resultados. Se para uma mesma escala $r$,
os valores de $g_{12}(r)$ diferirem dos de $g_{21}(r)$, então a relação espacial será
considerada assimétrica. Se além disso um dos valores for maior do que 1, esse resultado é um
sinal da ocorrência de facilitação.

Para avaliar a significância dos padrões encontrados, faremos simulações de Monte-Carlo nas
quais o padrão 1 é mantido fixo e o padrão 2 é distribuído aleatoriamente. 
Cada simulação gerará um valor de $g$. Após as simulções, construiremos um envelope de confiança contendo 95\% dos valores de
$g_{12}$ gerados. Desvios acima do envelope indicam associação espacial e abaixo,
dissociação. Para garantir a significância estatística, usaremos também estatísticas escalares
\citep{DiggleBook}. Se 
A distribuição do
padrão 2 será feita de acordo com dois modelos nulos diferentes. O primeiro modelo nulo é o
processo de Poisson homogêneo, no qual os pontos têm igual probabilidade de ocorrer em
qualquer posição da área de estudo durante as simulações. O segundo modelo nulo, usado para
distingüir a associação causada por filtragem ambiental da causada por facilitação, é um
processo de Poisson heterogêneo, no qual a probabilidade de ocorrência dos pontos varia no
espaço. Assim, enquanto no processo homogêneo a intensidade do processo, λ, é constante, no
processo heterogêneo os pontos são distribuídos proporcionalmente a uma função de
intensidade, λ(x,y). A função de intensidade será criada a partir da distribuição dos pontos
de cada espécie, de forma que ela será maior nas áreas onde houverem mais indivíduos daquela
espécie. Por isso, esse modelo nulo permitirá distingüir se as populações estão associadas
dentro das áreas de alta probabilidade de ocorrência de indivíduos (por exemplo, manchas de
ambiente propício).

Para comparar a distribuição espacial dos indivíduos com as variáveis ambientais, 
A associação das densidades com as variáveis ambientais será calculada por regressão
múltipla. Se o residual da regressão for grande e tiver estrutura, o mais provável é que isso
se deva a autocorrelação espacial dos indivíduos, resultado da agregação gerada por limitação
de dispersão; nesse caso usaremos {\em Spatial
EigenVector Mapping} (SEVM) para refinar a regressão. A análise SEVM cria variáveis virtuais
extras para representar o efeito da autocorrelação espacial, e separar esse efeito dos outros
fatores determinantes da densidade \citep{refsDiniz,Dormann2007methods}.

Como mencionado na introdução, para duas espécies
serem dispersas pelos mesmos dispersores, elas devem ter a mesma síndrome de dispersão, e a
mesma época de frutificação.
Usaremos análise circular para determinar a sazonalidade da frutificação de cada
espécie, e quais espécies frutificam nos mesmo períodos \citep{analisecircular}. 
Usaremos modelos lineares generalizados para medir a influência da proximidade de adultos da mesma
espécie na densidade de sementes em cada coletor \citep{Hardesty2002}. 
Verificaremos se existe associação espacial entre sementes de espécies diferentes usando a
função de correlação de densidades $g_{12}$, uma versão para densidades da função de
correlação de pares \citep{Fedriani2015}. 
Para os pares de espécies que apresentarem associação espacial de sementes, para verificar se
a associação das sementes é gerada pela associação dos adultos, usaremos
como modelo nulo uma distribuição de sementes gerada por uma distribuição heterogênea, usando
a distribuição dos adultos para determinar a função de densidade.
Se a associação das sementes for significativamente maior do que a previsão gerada por esse modelo nulo,
e se a síndrome de dispersão e a época de frutificação forem as mesmas, isso será evidência
de que a associação espacial é gerada pela dispersão pelos mesmos dispersores (codispersão).

\section{Cronograma}

	\begin{figure}[ht]
		\centering \includegraphics[width=15cm]{cronograma.png}
    \end{figure}

A qualificação no Programa de Pós-Graduação em Ecologia da UNICAMP está prevista para o final do
primeiro ano. Ao longo dos dois anos, serão feitos três relatórios (dois parciais e um final),
cursadas disciplinas de interesse e realizados estágios de docência. Também participarei de
congressos, onde apresentarei trabalhos resultantes deste projeto.

\section{Resultados esperados}
O presente estudo contribuirá para o entendimento dos fatores que determinam a distribuição espacial
das espécies arbóreas em florestas tropicais, e para o conhecimento das interações ecológicas entre
espécies de árvores, resultando em última instância em avanço do conhecimento sobre a diversidade
das florestas tropicais. Durante o projeto serão redigidos dois artigos, e o trabalho será apresentado
em congressos.



    \bibliographystyle{apalike}
    \bibliography{tcc}
\end{document}
